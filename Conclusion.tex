\chapter{Conclusion}

In the past 4 months of the internship, we were able to test the IIDEAA framework extensively, thus detected some problems around it, regarding many aspects such as the core code base, run-time issue, conflicting dependencies between run-time environment and some of the projects. From the results obtained from applying the framework to some of the AxBench's benchmark, we can further validate that choosing an suitable mutation operator is crucial when trying to utilize Approximate Computing. In addition, FANN has proven to be a Artificial Neural Network library that is compatible with IIDEAA to a certain extend. Nevertheless, further experiments might needed to see whether IIDEAA can apply Approximate Computing to everything in the library. \\
~\\
In the 2 upcoming months of the internship, we are planning to develop a tool that can generate separate variants of the mutated source code based on the result of Bellerophon. This tool will be helpful when manually editting the source code becomes infeasible due to either having large amount of acceptable results or high number of operators to modify. \\
~\\
In the future, we hope to cover all of the fore-mentioned drawbacks of IIDEAA, especially moving IIDEAA to a better run-time environment with no conflicting dependencies as well as having the latest LLVM version. Developing a new Approxmiate Operator is also desired as we would like to have more versatility within IIDEAA.\\