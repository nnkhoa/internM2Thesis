\chapter{Introduction}

\section{Context and Motivation}
%Very brief introduction on Artificial Neural Network

Deep Learning is an approach used in the field of Artificial Intelligence and Machine Learning to help computers solve problems that are easy and intuitive for human to execute yet extremely difficult to give a formal explanation, such as object recognition. The technique centers around assisting computers in understanding the most fundamental, simplest notions of the task, building experiences and working its way through a hierarchy of concepts with increasing complexity. One of the more basic and earlier algorithms in Deep Learning is Artificial Neural Network, or Neural Network for short, which is inspired by the biological brain, thus having interconnected neurons that can receive input and produce real-value output \cite{Goodfellow-et-al-2016}. \\
~\\
%Current state of Neural Network
As of the time of writing, Artificial Neural Network, and Deep Learning in general, is becoming a trending technique in the field of Machine Learning and Artificial Intelligence, as it is being widely used and researched by many organizations. It has produced great results in complex tasks such as image recognition \cite{Krizhevsky:2012:ICD:2999134.2999257}, natural language processing \cite{recent-advances-in-deep-learning-for-speech-research-at-microsoft}, or even playing sophisticated games such as the Go game \cite{GoGame}. \\ 
~\\
%NN has disadvantages as consumes time/electricity while demands high-end hardware
While having great achievements in the past years, Deep Learning still suffers from significant drawbacks. One of its major disadvantages is being quite resource-demanding, requiring high amount of power, time and hardware capability. For example, for AlphaGo to be able to obtain its result, it cost the team 4 to 6 weeks of training the model, required 2000 CPUs and 250 GPUs which consumed 600kW of electricity. These were huge numbers, especially when compare to a 20W brain power of a Go player \cite{GoGame}. Due to this problem, researches have been conducted, both from industrial and academic side, to find an equilibrium between cost and performance. \\
~\\
%AxC might be a solution but required automated framework to be efficient
Because Artificial Neural Network has shown itself to be inherently resilient to insignificant errors, being able to use lower arithmetic precision while not hurting the overall result of the network \cite{DBLP:journals/corr/SungSH15}, Approximate Computing is a promising solution to the previously stated problem. However, due to network's high complexity, it requires a lot of time and effort to being able to explore all possible Approximate Computing variants of the neural network. Hence, an automatic framework is essential for having proficiency in finding optimal Approximated solutions. \\

\section{Internship's Objective}
In this give context, the internship at Laboratoire d'Informatique, de Robotique et de Microélectronique de Montpellier (LIRMM) focuses on the following goals: ~\\
\begin{itemize}
	\item Test and debug to improve IIDEAA, an automated framework designed to explore Approximate Computing variations of the target program.
	\item Check the framework compatibility with some C/C++ neural network library and application.
	\item Develop an automatic source-to-source mutation that will automatically generate Approximate Computing variants as the framework right now is only able to explore and test. 
\end{itemize}

\section{Host Institution - LIRMM}

Founded in 1992, LIRMM, shorts for Laboratoire d'Informatique, de Robotique et de Microélectronique de Montpellier, is a joint research laboratory between the Centre National de la Recherche Scientifique(CNRS) and the University of Montpellier. The lab has gathered many researchers from different countries, such as France, Italy, China, Brazil, and from differents institute, namely CNRS, University of Montpellier, Institut National de Recherche en Informatique et en Automatique(INRIA). \cite{LIRMM}.~\\
~\\
%LIRMM image
\begin{figure}[h]
\includegraphics[angle=0,width=11cm]{lirmm.png}
\centering
\caption{LIRMM Official Logo}
\end{figure}
~\\
Researches at LIRMM concentrate on computer science, microelectronics, and robotics that cover the following interests: Design and verify integration process, mobile and communication systems, agent-based modeling on complex systems, and studies in algorithm, bioinformatic, human-machine interactions, robotics. Research's results usually have been applied in various fields, for example, biology, telecommunication, healthcare. They are aslo used internally in the lab \cite{LIRMM}. ~\\ 
\vspace*{3cm}


\section{Report organization}

The report is organized as follow: Excluding the Introduction chapter, there are 5 chapters. The \nth{2} chapter discusses background knowledge related to Approximate Computing and one of its method, Precision Scaling. In addition, this section will also take into account the IIDEAA framework and some State of the Art that related to developing simillar framework. Chapter 3 describes the contributions made during the internship period. Testing and results are shown in the \nth{4} chapter. Last chapter will conclude the report with summary of the internship's current state, limitations faced, and possible future work. 